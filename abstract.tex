% Created 2021-11-03 Wed 13:36
% Intended LaTeX compiler: pdflatex
\documentclass[11pt]{article}
\usepackage[utf8]{inputenc}
\usepackage[T1]{fontenc}
\usepackage{graphicx}
\usepackage{grffile}
\usepackage{longtable}
\usepackage{wrapfig}
\usepackage{rotating}
\usepackage[normalem]{ulem}
\usepackage{amsmath}
\usepackage{textcomp}
\usepackage{amssymb}
\usepackage{capt-of}
\usepackage{hyperref}
\usepackage{minted}
\author{Jakob Klemm}
\date{}
\title{Abstract}
\hypersetup{
 pdfauthor={Jakob Klemm},
 pdftitle={Abstract},
 pdfkeywords={},
 pdfsubject={},
 pdfcreator={Emacs 28.0.50 (Org mode 9.4.4)}, 
 pdflang={English}}
\begin{document}

\noindent Wir leben in einer bequemen Welt. Das wollen wir wenigstens
glauben.In Wahrheit leben wir aber in einer idyllischen Illusion. Die
unglaublichen Veränderungen unserer Gesellschaft und unseres Alltags
hätte sich vor 30 Jahren niemand vorstellen können. Die
Geschwindigkeit der Fortschritte hat noch nie gesehene Mengen an
Wohlstand und Reichtum geschaffen. Doch im Rausch des Wandels wurden
gewisse Entscheidungen, manche bewusst und böswillig, andere aus Not,
getroffen, welche uns in unserer modernen, vom Internet vollständig
abhängigen, Gesellschaft in eine prekäre Situation bringen. Die
tiefsten Fundamente der Welt, wie wir sie kennen, sind dem totalen
Zusammenbruch gefährlich nahe, und was das System am Laufen hält sind
oftmals monopolistische Megaunternehmen und ihre ungewählten CEO's,
die nur von der Weltherrschaft träumen.\\

\noindent Es ist offensichtlich, dass eine Alternative her muss. Aber
während verschiedenste Projekte bereits grosse Fortschritte in den
Bereichen der dezentralen Kommunikation und Datenspeicherung machen,
bleiben oftmals die Nutzer aus. Denn trotz aller Innovation und aller
Vorteile geht es hier um äusserst abstrakte, komplexe Themen, welche
Unmengen an Vorwissen und Interesse benötigen, um sie genügend zu
verstehen. \texttt{Project Orion} versucht genau da anzusetzen: Nebst einem
dezentralen Kommunikationssystem sollen auch verschiedene, praktisch
einsetzbare Anwendungen die Prinzipien dezentraler Systeme in die
Hände der Endnutzer bringen.
\end{document}
