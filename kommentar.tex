% Created 2021-09-07 Di 10:37
% Intended LaTeX compiler: pdflatex
\documentclass[11pt]{report}
\usepackage[utf8]{inputenc}
\usepackage[T1]{fontenc}
\usepackage{graphicx}
\usepackage{grffile}
\usepackage{longtable}
\usepackage{wrapfig}
\usepackage{rotating}
\usepackage[normalem]{ulem}
\usepackage{amsmath}
\usepackage{textcomp}
\usepackage{amssymb}
\usepackage{capt-of}
\usepackage{hyperref}
\usepackage{minted}
\usepackage[utf8]{inputenc}
\usepackage[dvipsnames]{xcolor}
\usepackage{tikz}
\usepackage{pdfpages}
\usepackage[, germanb]{babel}
\usepackage{listings}
\usepackage[]{babel}
\usepackage[dvipsnames]{xcolor}
\usepackage{courier}
\usepackage{listings}
\usepackage{textcomp}
\usepackage{gensymb}
\author{Jakob Klemm, Dominik Keller}
\date{}
\title{Projekt Orion}
\hypersetup{
 pdfauthor={Jakob Klemm, Dominik Keller},
 pdftitle={Projekt Orion},
 pdfkeywords={},
 pdfsubject={},
 pdfcreator={Emacs 28.0.50 (Org mode 9.4.4)}, 
 pdflang={Germanb}}
\begin{document}

\maketitle
\tableofcontents

\newpage

\textbf{Vorwort}\\
Die schriftliche Komponente der Maturaarbeit \texttt{Projekt Orion} besteht aus
drei verschiedenen Teilen. Da die behandelten Themen äusserst komplex
und umfangreich sind, verlangen verschiedene Abschnitte der Arbeit
verschiedenes Vorwissen und einen verschiedenen Zeitaufwand. Deswegen
wurde die schriftliche Komponente in drei Subkomponenten aufgeteilt,
wobei sie nach technischem Detailgrad sortiert sind. Wer nur ein
oberflächliches Verständnis über die Arbeiten und eine Analyse des
Umfelds will, ohne dabei zu technisch zu werden, muss nicht über den
Umfang dieses Dokuments hinaus. Aber für vollständigen Einblick in die
Errungenschaften und Konzepte muss mit einem grösseren Aufwand
gerechnet werden.
\begin{itemize}
\item Schriftlicher Kommentar: In diesem Dokument hier findet sich eine
klassische Besprechung der Arbeit. Begonnen mit einer Zielsetzung
und Besprechung verschiedener Projekte, bis zur Analyse des Produkts
und einem Ausblick in die Zukunft gibt dieses Dokument einen guten,
aber oberflächlichen Einblick in das \texttt{Projekt Orion}. Natürlich wird
besonders bei der Analyse der existierenden Projekten und Darlegung
des Konzepts gewisses technisches Know-How benötigt, aber es wurde
versucht, alle Fachbegriffe zu umschreiben oder zu erklären. Wer nur
über die Vision und den aktuellen Stand wissen will muss nicht über
dieses Dokument hinaus, aber verschiedene Konzepte und nahezu die
gesamte technische Umsetzung befinden sind nicht in diesem Dokument.
\item Dokumentation: In dieser alleinstehenden Dokumentation, welche im
Detailgrad zwischen dem schriftlichen Kommentar und der
Code-Dokumentation steht, werden die Konzepte und Ideen besprochen.
Wer die Entstehung und aktuelle Form der Komponenten genauer
verstehen will, oder wer von den umgesetzten Funktionen profitieren
will, sollte die Dokumentation durcharbeiten. Das Dokument ist eher
umfangreich, es kann aber auch gut als eine Art Nachschlagewerk
verwendet werden.
\item Code: Neben der Dokumentation des Projekts und der Konzepte,
existiert eine weitere Form der Dokumentation. Nahezu jede Funktion,
jedes Modul und jedes Objekt über die verschiedenen \emph{Crates} sind
dokumentiert. Diese Dokumentationen lassen sich nicht in einem
klassisch strukturierten Dokument finden. Stattdessen ist die
Code-Dokumentation online über automatisch generierte
Rust-Dokumentation zu finden. Die Seiten mögen anfangs etwas
unübersichtlich wirken, wer aber den Code von \texttt{Projekt Orion}
verwenden will wird sich dort gut zurecht finden.
\end{itemize}

\newpage  
\begin{ABSTRACT}
TODO: Abstract
\end{ABSTRACT}
\newpage

\part{Vision}
\label{sec:org0e713b4}
\chapter{Grenzen}
\label{sec:org0693d08}
\chapter{Inhalte}
\label{sec:orged7bd79}
\chapter{Routing}
\label{sec:org5179e0a}
\chapter{Zentralisierung}
\label{sec:orgc5e407b}
\chapter{Orion}
\label{sec:orgaf50978}
\part{Projekte}
\label{sec:org961ef9c}
\chapter{Kademlia}
\label{sec:orgba7c203}
\chapter{CJDNS}
\label{sec:org4533bee}
\chapter{IPFS}
\label{sec:orgcdae0a0}
\chapter{Tox}
\label{sec:org38f4251}
\part{Konzept}
\label{sec:org28b70d4}
\part{Prozess}
\label{sec:orgf578873}
\chapter{Modularität}
\label{sec:orgbd2d6c9}
\chapter{Präsentation}
\label{sec:org9383537}
\part{Produkt}
\label{sec:orgb7ecf9e}
\part{Ausblick}
\label{sec:orgf2db332}
\chapter{Verifizierung}
\label{sec:org1fb9006}
\chapter{Balance}
\label{sec:orgb5847ab}
\end{document}
