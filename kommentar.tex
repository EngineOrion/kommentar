% Created 2021-10-31 So 10:19
% Intended LaTeX compiler: pdflatex
\documentclass[11pt]{article}
\usepackage[utf8]{inputenc}
\usepackage[T1]{fontenc}
\usepackage{graphicx}
\usepackage{grffile}
\usepackage{longtable}
\usepackage{wrapfig}
\usepackage{rotating}
\usepackage[normalem]{ulem}
\usepackage{amsmath}
\usepackage{textcomp}
\usepackage{amssymb}
\usepackage{capt-of}
\usepackage{hyperref}
\usepackage{minted}
\usepackage[utf8]{inputenc}
\usepackage[dvipsnames]{xcolor}
\usepackage{tikz}
\usepackage{pdfpages}
\usepackage[, germanb]{babel}
\usepackage{listings}
\usepackage[]{babel}
\usepackage[dvipsnames]{xcolor}
\usepackage{courier}
\usepackage{listings}
\usepackage{textcomp}
\usepackage{gensymb}
\author{Jakob Klemm, Dominik Keller}
\date{}
\title{Project Orion}
\hypersetup{
 pdfauthor={Jakob Klemm, Dominik Keller},
 pdftitle={Project Orion},
 pdfkeywords={},
 pdfsubject={},
 pdfcreator={Emacs 28.0.50 (Org mode 9.4.4)}, 
 pdflang={Germanb}}
\begin{document}

\maketitle
\tableofcontents

\newpage  
\begin{ABSTRACT}
TODO: Abstract
\end{ABSTRACT}
\newpage

\textbf{Vorwort}\\
Die schriftliche Komponente der Maturaarbeit \texttt{Projekt Orion} besteht aus
drei verschiedenen Teilen. Da die behandelten Themen äusserst komplex
und umfangreich sind, verlangen verschiedene Abschnitte der Arbeit
verschiedenes Vorwissen und einen verschiedenen Zeitaufwand. Deswegen
wurde die schriftliche Komponente in drei Subkomponenten aufgeteilt,
wobei sie nach technischem Detailgrad sortiert sind. Wer nur ein
oberflächliches Verständnis über die Arbeiten und eine Analyse des
Umfelds will, ohne dabei zu technisch zu werden, muss nicht über den
Umfang dieses Dokuments hinaus. Aber für vollständigen Einblick in die
Errungenschaften und Konzepte muss mit einem grösseren Aufwand
gerechnet werden.
\begin{itemize}
\item Schriftlicher Kommentar: In diesem Dokument hier findet sich eine
klassische Besprechung der Arbeit. Begonnen mit einer Zielsetzung
und Besprechung verschiedener Projekte, bis zur Analyse des Produkts
und einem Ausblick in die Zukunft gibt dieses Dokument einen guten,
aber oberflächlichen Einblick in das \texttt{Projekt Orion}. Natürlich wird
besonders bei der Analyse der existierenden Projekten und Darlegung
des Konzepts gewisses technisches Know-How benötigt, aber es wurde
versucht, alle Fachbegriffe zu umschreiben oder zu erklären. Wer nur
über die Vision und den aktuellen Stand wissen will muss nicht über
dieses Dokument hinaus, aber verschiedene Konzepte und nahezu die
gesamte technische Umsetzung befinden sind nicht in diesem Dokument.
\item Dokumentation: In dieser alleinstehenden Dokumentation, welche im
Detailgrad zwischen dem schriftlichen Kommentar und der
Code-Dokumentation steht, werden die Konzepte und Ideen besprochen.
Wer die Entstehung und aktuelle Form der Komponenten genauer
verstehen will, oder wer von den umgesetzten Funktionen profitieren
will, sollte die Dokumentation durcharbeiten. Das Dokument ist eher
umfangreich, es kann aber auch gut als eine Art Nachschlagewerk
verwendet werden.
\item Code: Neben der Dokumentation des Projekts und der Konzepte,
existiert eine weitere Form der Dokumentation. Nahezu jede Funktion,
jedes Modul und jedes Objekt über die verschiedenen \emph{Crates} sind
dokumentiert. Diese Dokumentationen lassen sich nicht in einem
klassisch strukturierten Dokument finden. Stattdessen ist die
Code-Dokumentation online über automatisch generierte
Rust-Dokumentation zu finden. Die Seiten mögen anfangs etwas
unübersichtlich wirken, wer aber den Code von \texttt{Projekt Orion}
verwenden will wird sich dort gut zurecht finden.
\end{itemize}
\newpage

\section{Vision}
\label{sec:org553c95a}
\begin{center}
\begin{enumerate}
\item Oktober 1964 - UCLA an Stanford: LO
\end{enumerate}
\end{center}
\emph{1964} rechnete niemand mit der fundamentalen Änderung unserer Existenz
und Lebensweise, die mit dieser einfachen Nachricht in Bewegung
gebracht wurde. Eigentlich hätte die erste Nachricht über das \texttt{ARPANET}
im Jahre 1964 \texttt{LOGIN} heissen sollen, doch das Netzwerk stürzte nach nur
zwei Buchstaben ab. Ob dies als schlechtes Omen für die Zukunft hätte
gewertet werden sollen bleibt eine ungeklärte Frage. Aber das Internet
ist hier und es ist so dominant wie noch nie zuvor. Jetzt ist es in
der Verantwortung jeder neuen Generation auf diesem Planeten, mit den
unglaublichen Möglichkeiten richtig umzugehen und die Vielzahl an
bevorstehenden Katastrophen und Gefahren zu navigieren.\\

\noindent Ohne das Internet wäre die Welt wie wir sie kennen nicht
möglich. Unsere Arbeit, Kommunikation und unser Entertainment sind
nicht einfach nur abhängig von der enormen Interkonnektivität des
Internets, ohne sie würden ganze Industrien und Bereiche unser
Gesellschaft gar nicht erst existieren. Das Internet hatte einen
selbstverstärkenden Effekt auf sein eigenes Wachstum. Der um ein
Vielfaches schnellere Datenaustausch und die enorme Interkonnektivität
führten dazu, dass jede neue Innovation und jede neue Plattform im
Internet noch schneller noch mehr User erreichte und auf immer
unvorstellbarere Grössen anwuchs.\\

\noindent Das ist ja grundsätzlich nichts Schlechtes. Das Internet hat
eine unvorstellbare Menge an Vermögen, Geschwindigkeit und
Bequemlichkeit für uns alle geschaffen und wir haben unsere
Gesellschaftsordnung daran ausgerichtet. Aber man muss sich fragen, ob
wir manche der Schritte nicht doch überstürzt haben. Im Namen des
Wachstums und aus \texttt{FOMO} (\emph{Fear Of Missing Out}) wurden Technologien für
die Massen zugänglich, die eigentlich nie für solche Grössenordnungen
entwickelt wurden. Denn sobald die immer höheren Erwartungen an teils
unglaublich fragile Systeme nicht mehr erfüllt werden, kommt es
schnell zur Katastrophe. Und durch unsere Abhängigkeit von diesen
Systemen steht bei einem solchen Szenario nicht nur der Untergang
einiger Produkte oder einzelner Firmen bevor, nein, es könnte zum
Kollaps ganzer Länder oder Gesellschaften kommen.\\

\noindent Egal wie sicher und zuverlässig unsere \emph{öffentliche}
Infrastruktur auch scheinen mag, es lassen sich doch schnell Risse im
System erkennen. Nicht nur an der Oberfläche, sondern auch im Herzen
unseres digitalen Leben gibt es Probleme. Oftmals handelt es sich
dabei nicht um \emph{Kleinigkeiten}, \emph{Meinungsverschiedenheiten} oder
\emph{Kontroversen}, sondern um physikalische Grenzen, grundlegende
Designfehler und das vielseitige Versagen der involvierten Parteien.\\

\noindent In den nächsten Kapiteln sollen einige dieser zentralen
Probleme besprochen werden. Dabei soll versucht werden nicht nur die
fehlerhaften Implementierungen zu erklären, sondern auch die dadurch
entstandenen Probleme in Verbindung mit unseren täglichen
Interaktionen und Verwendungen des Internets zu bringen. In einem
nächsten Schritt soll dann eine Lösung besprochen werden: ein System,
mit welchem sich möglichst viele der grössten Probleme lösen lassen,
und welches tatsächlich praktischen Nutzen bietet.\\
\subsection{Adressen}
\label{sec:org24c5c77}
Das Internet erlaubt einfache, standardisierte Kommunikation zwischen
Geräten aller Art. Egal welche Funktion oder Form sie auch haben
mögen, es braucht nicht viel um ein Gerät mit dem Internet zu
verbinden. Nebst den benötigten Protokollen, hauptsächlich \texttt{TCP} und
\texttt{UDP} wird eine \texttt{IP-Addresse} als eindeutige Identifikation benötigt.
Während vor dreisig Jahren wunderbare Systeme und Standards geschaffen
wurden, welche seither die Welt grundlegend verändert haben, so gibt
es doch einige fundamentale Probleme und Limitierungen, welche zu
immer grösseren Problemen für unsere verbundene Welt werden.
\subsubsection{IP-V4}
\label{sec:org097a22f}
\noindent In der Geschichte der Menschheit haben wir aus vielen
verschiedenen Gründen Krieg geführt. Für Wasser, Nahrung, Öl, Frieden
oder Freiheit in den Krieg zu ziehen, scheint zu einer fernen Welt zu
gehören. Aber auch wenn diese grundlegenden Verlangen gedeckt sind,
werden schon bald neue Nöte aufkommen. Während \emph{Daten} oft als Gold
des 21. Jahrhunderts bezeichnet werden, gibt es noch eine andere
Ressource, deren Vorräte wir immer schneller erschöpfen. \\

\noindent \(4'294'967'296\). So viele \texttt{IP-V4}-Adressen wird es jemals
geben. \texttt{IP-V4}-Adressen werden für jedes Gerät benötigt, das im Internet
kommunizieren will und dienen zur eindeutigen Identifizierung. Aktuell
wird die vierte Version (\texttt{V4}) verwendet. In einer Wirtschaft, in der
unendliches Wachstum als letzte absolute Wahrheit geblieben ist, kann
ein solch hartes Limit verheerende Folgen haben. Besonders wenn die
limitierte Ressource so unendlich zentral für unser aller Leben ist,
wie nichts anderes. Mit \texttt{IP-V6} wird zurzeit eine Alternative angeboten,
die solche Limitierungen nicht hat. Aber der Wechsel ist eine
freiwillige Entscheidung, für die nicht nur alle Betroffenen bereit
sein müssen, sondern für die auch jede einzelne involvierte Komponente
diese neue Technologie unterstützen müssen.\\

Für jeden einzelnen kann dies verschiedene Konsequenzen haben:
\begin{itemize}
\item Die Preise der Internetanbieter und Mobilfunkabonnemente wird
wahrscheinlich langfristig steigen, sobald die ehöhten Kosten für
neue Addressen bis zum Endnutzer durchsickern.
\item Ein aufwengdiger technologischer Wandel wird langfristig von Nöten
sein, welchen jeden einzelnen dazu zwingt auf neue Standards
umzusteigen. Eine solche Umstellung wird den häufigen Problemen
grossflächiger technischer Umstellungen nicht ausweichen können.
\end{itemize}
\subsubsection{Routing}
\label{sec:orge60f8f1}
Freiheit und Unabhängigkeit sind menschlich. Es darf niemals bestraft
werden, nach diesen fundamentalen Rechten zu streben. Und doch führt
das egoistische Streben nach Freiheit zu Problemen, oftmals allerdings
nicht für die nach Freiheit Strebenden.\\

\noindent Genau diese Situation findet man im aktuellen Konflikt um
die Grösse von \emph{Address-Abschnitten} vor. Um dieses Problem richtig zu
verstehen, muss als erstes die Funktion der \emph{Zentralrouter} und der
globalen Netzwerkinfrastruktur erklärt werden:\\

\noindent Jedes Gerät im Internet ist über Kabel oder Funk mit jedem
anderen Gerät verbunden. Da das Internet aus einer Vielzahl von
Geräten besteht, wäre es unmöglich, diese direkt miteinander zu
verbinden. Daher lässt sich das Internet besser als \emph{umgekehrte
Baum-Struktur} vorstellen:
\begin{itemize}
\item Ganz unten finden sich die Blätter, die Abschlusspunkte der
Struktur. Sie stellen die \emph{Endnutzergeräte} dar. Jeder Server, PC und
jedes \texttt{iPhone}. Hier ist es auch wichtig festzustellen, dass es in
dieser Ansicht des Internets keine magische \emph{Cloud} oder ferne Server
und Rechenzentren gibt. Aus der Sicht des Netzwerks sind alle
Endpunkte gleich, auch wenn manche für Konsumenten als \emph{Server}
gelten.
\item Die Verzweigungen und Knotenpunkte über den Blättern, dort wo sich
Äste aufteilen, stellen \emph{Router} und Switches dar. Hier geht es
allerdings nicht um Geräte, die sich in einem persönlichen Setup
oder einem normalen Haushalt finden. Mit Switches sind die
Knotenpunkte (\texttt{POP-Switches}) der Internet-Anbieter gemeint. Diese
teilen eingehende Datenströme auf und leiten die richtigen Daten
über die richtigen Leitungen.
\item Ganz oben findet sich der Stamm. Während ein normaler Baum natürlich
nur einen Stamm hat, finden sich in der Infrastruktur des Internets
aus Zuverlässigkeitsgründen mehrere. Von diesen \texttt{Zentralroutern} gibt
es weltweit nur eine Handvoll und sie sind der Grund für das
Problem.
\end{itemize}

\noindent Die \texttt{Zentralrouter} kümmern sich nicht um einzelne Adressen,
sondern um Abschnitte von Adressen, auch \texttt{Address Spaces} genannt. An
den zentralen Knotenpunkten geht es also nicht um einzelne Server oder
Geräte, zu dem etwas gesendet werden muss, stattdessen wird eher
entschieden, ob gewisse Daten beispielsweise von Frankfurt aus nach
Ost- oder Westeuropa geschickt werden müssen.\\

\noindent Im Laufe der Jahre wurden die grossen Abschnitte von
Adressen aber immer weiter aufgeteilt. Internet-Anbieter und grosse
Firmen können diese Abschnitte untereinander verkaufen und aufteilen.
Und jede Firma will natürlich ihren eigenen Abschnitt, ihren eigenen
\texttt{Address Space}. Für die Firmen hat dies viele Vorteile, beispielsweise
müssen weniger Parteien beim Finden des korrekten Abschnitts
involviert sein. Aber für die \texttt{Zentralrouter} bedeutet es eine immer
grössere Datenbank an Zuweisungen. Dieses Problem geht so weit, dass
die grossen \emph{Routingtables} inzwischen das physikalische Limit
erreichen, was ein einzelner Router verarbeiten kann.\\
\subsection{Zentralisierung}
\label{sec:orgd93ee33}
\noindent Die Macht in den Händen einiger weniger Kapitalisten und
internationaler Unternehmen ist unvorstellbar gross. Einige wenige
CEO's, welche nie gewählt, überprüft oder zur Rede gestellt wurden,
sind in voller Kontrolle unserer Leben. Egal welcher politischen,
wirtschaftlichen oder gesellschaftlichen Ideologie jemand auch folgt,
eine solche Abhängigkeit wirft gewisse Fragen und Probleme auf.\\

\noindent Aber neben den ideologischen Fragen und Sicherheitsbedenken
gibt es auch noch sehr praktische Probleme in der Art, wie moderne
Internet-Dienste implementiert sind.
\subsubsection{Datenschutz}
\label{sec:org5776f18}
\begin{center}
\emph{Wenn man nicht für etwas zahlt, ist man das Produkt.}
\end{center}
Nach dieser Idee ist man für ziemlich viele Firmen ein Produkt. Doch
leider muss man realisieren, dass man selbst bei kostenpflichtigen
Diensten als Produkt gesehen wird. Denn das Internet hat einen neuen
Rohstoff zur Welt gebracht. Wer viele Daten über Menschen besitzt,
bekommt binnen kürzester Zeit Macht.\\

\noindent In ihrer einfachsten Funktion werden Daten für
personalisierte Werbung eingesetzt. Damit lassen sich Werbungen
zielgerichtet an Konsumenten schicken und der Umsatz, sowohl für
Firmen als auch für Anbieter, optimieren.\\

\noindent Werbung ist mächtig und hat einen grossen Einfluss auf den
Markt. Aber damit lassen sich lediglich Konsumenten zu Käufen
überzeugen oder davon abbringen. Wenn man dies mit dem tatsächlichen
Potential in diesen Daten vergleicht, merkt man schnell, wie viel noch
möglich ist. Denn die Daten die sich täglich über uns im Internet
anhäufen, zeigen mehr als unser Kaufverhalten. Von
Echtzeit-Positionsupdates, Anrufe und Suchanfragen bis hin zu privaten
Chats und unseren tiefsten Geheimnissen, sind wir meist überraschend
unvorsichtig im Umgang mit digitalen Werkzeugen.\\

\noindent Während man davon ausgehen muss, dass Firmen, deren
Haupteinnahmequelle Werbungen ist, unsere Daten sammeln und verkaufen,
gibt es eine Vielzahl an anderen Firmen, die ebenfalls unsere Daten
sammeln, obwohl man von den meisten dieser Firmen noch nie gehört hat.
Die Liste der potentiellen Mithörer bei unseren digitalen
Unterhaltungen ist nahezu unendlich: Internet-Anbieter,
DNS-Dienstleister, CDN-Anbieter, Ad-Insertion-Systeme,
Analytics-Tools, Knotenpunkte \& Datencenter, Browser, Betriebssysteme,
\ldots{}.\\

\noindent Aus dieser Tatsache heraus lassen sich zwei zentrale
Probleme formulieren:
\begin{itemize}
\item Selbst für die einfachsten Anfragen im Internet sind wir von einer
Vielzahl von Firmen und Systemen abhängig. Dieses Problem wird noch
etwas genauer im Abschnitt \hyperref[sec:orge7d426d]{Abhängigkeit} besprochen.
\item Wir haben weder ein Verständnis von den involvierten Parteien noch
die Bereitschaft, Bequemlichkeit dafür aufzugeben.
\end{itemize}
\subsubsection{Abhängigkeit}
\label{sec:orge7d426d}
In einem fiktionalen Szenario\footnote{Tom Scott: Single Point of Failure
\url{https://youtu.be/y4GB\_NDU43Q}, heruntergeladen am 24.05.2020.} erklärt \emph{Tom Scott} auf seinem
YouTube-Kanal was passieren könnte, wenn eine einzelne
Sicherheitsfunktion beim Internetgiganten \texttt{Google} fehlschlagen würde.
In einem solchen Fall ist es natürlich logisch, dass es zu Problemen
bei den verschiedensten \texttt{Google}-Diensten kommen würde. Aber schnell
realisiert man, auf wie vielen Seiten Nutzer die \emph{Sign-In with Google}
Funktion benutzen. Und dann braucht es nur eine böswillige Person um
den Administrator-Account anderer Dienste und Seiten zu öffnen,
wodurch die Menge an Sicherheitsproblemen exponentiell steigt.\\

\noindent Aber es muss nicht immer etwas schief gehen, um die Probleme
zu erkennen. Sei es politische Zensur, \emph{Right to Repair} oder \emph{Net
Neutralität}, die grossen Fragen unserer digitalen Zeit sind so
relevant wie noch nie.\\

\noindent Während die enorme Abhängigkeit als solche bereits eine
Katastrophe am Horizont erkennen lässt, gibt es noch ein konkreteres
Problem: Den Nutzern (\emph{den Abhängigen}) ist ihre Abhängigkeit nicht
bewusst. Wenn sie sich ihren Alltag ohne \texttt{Google} oder \texttt{Facebook}
vorstellen, denken sich viele nicht viel darunter. Weniger \emph{lustige
Quizfragen} oder Bilder von Haustieren, aber was könnte den schon
wirklich Schlimmes passieren?\\

\noindent Während es verständlich ist, dass das Benutzen von \texttt{Google}
natürlich von \texttt{Google} abhängig ist, so versteht kaum jemand, wie viel
unserer täglichen Aktivitäten von Diensten und Firmen abhängen, die
selbst wieder von \texttt{Google} abhängig sind. Seien es die \texttt{Facebook}-Server,
durch welche keine Whatsapp-Nachrichten geschickt werden konnten, oder
die fehlerhafte Konfiguration bei \texttt{Google}, durch welche manche Kunden
die Temperatur ihrer Wohnungen auf ihren Nest Geräten nicht mehr
anpassen konnten\footnote{Fastcompany: Google outage, heruntergeladen am 24.10.2021.
\url{https://www.fastcompany.com/90358396/that-major-google-outage-meant-some-nest-users-couldnt-unlock-doors-or-use-the-ac}}, das Netz aus internen Verbindungen zwischen
Firmen ist komplex und undurchschaubar, und nicht nur für die
Entnutzer, da oftmals die Firmen selbst von kleinsten Problemen
anderer Dieste überrascht werden können. Der wirtschaftliche Schaden
solcher Ausfälle sind unvorstellbar, aber noch wichtiger muss die
zerstörende Wirkung dieser auf unvorhergesehenen, entfernten Problemen
auf millionen von Meschen bedacht werden.
\subsection{Komplexität}
\label{sec:orga843028}
In diesem Abschnitt soll noch kurz die unglaubliche Komplexität
angesprochen werden, welche die häutige Web-Entwicklung mit sich
bringt. Natürlich existieren automatisierte Dienste und Anbieter, die
den Prozess vereinfachen. Wer aber Wert auf seine Privatsphäre und auf
die Verwendung von open-source Software legt, muss sich um vieles
selbst kümmern. Nicht nur die Auswahl an verschiedenen Programmen kann
erschlagend wirken, sondern der Fakt, dass diese untereinander
kompatibel sein müssen. Zwar reden wir oft von einem Webserver,
allerdings sind es tatsächlich viele verschiedene Programme, die alle
fehlerfrei miteinander interagieren müssen, um Resultate zu liefern.
Dies kann den Einstieg schwerer machen, oder, in gefährlicheren Fällen
kann es dazu führen, dass Sicherheit und Datenschutz aus Zeit- oder
Komplexitätsgründen weggelassen oder vernachlässig werden.\\

\noindent Dabei geht es oben nur um \emph{klassische} Webseiten oder
Webserver. Die Welt der dezentralen Technologien ist im vergleich dazu
wie der wilde Westen, ohne Standards, ohne Kompatiblität oder
Regelungen. Dies führt dazu, dass es zwar für gewisse Anwendungen
speziell entwickelte Netzwerke gibt, diese allerdings kaum allgeimen
einsetzbar sind.
\subsection{Präsentation}
\label{sec:org9e95d8b}
Ein weiteres Problem, dass es zu berücksichtigen gibt, ist die Frage,
wie man die hier behandelten Probleme technisch nicht versierten
Personen erklären kann. Tatsächlich sind sowohl die besprochenen
Probleme, als auch deren Lösungsansätze nicht nur abstrakt, sondern
dazu noch Teil einer kleinen Nische in der Welt der Informatik.
Manche der obigen Probele wurden bereits von anderen Applikationen
zumindest teilweise behandelt, diese haben aber oftmals das Problem,
dass sie viel Fachwissen und Aufwand benötigen, um sie effizient und
sicher einzusetzen.
\section{Prozess}
\label{sec:org3cd0912}
\subsection{Modularität}
\label{sec:org6fecbcb}
Tatsächlich beschreibt dieses Dokument bereits den zweiten Versuch,
die zweite Iteration eines dezentralen Datensystems. Da während dieser
ersten Entwicklungsphase viele Lektionen gelernt wurden, ist es
wichtig die Ideen und die Umsetzung genau zu analysieren. Zwar
unterscheiden sich die Ziele und Methoden der beiden Ansätze stark,
gewisse Konzepte und einige Programme lassen sich für die aktuelle
Zielsetzung genau übernehmen.\\

\noindent Als erstes ist es wichtig, die Zielsetzung des Systems,
welches hier einfach als “Modularer Ansatz” bezeichnet wird, zu
verstehen und die damit entstandenen Probleme genau festzuhalten.
\begin{itemize}
\item Modularität \\
Wie der Name bereits verrät, ging es in erster Linie um die
Modularität. Ziel war also eine Methode zur standardisierten
Kommunikation, durch welche dann beliebige Komponenten an ein
grösseres System angeschlossen werden können. Mit einigen
vorgegebenen Komponenten, die Funktionen wie das dezentrale Routing
und lokales Routing abdecken, können Nutzer für ihre
Anwendungszwecke passende Programme integrieren.
\item Offenheit \\
Sobald man den Nutzern die Möglichkeit geben will, das System selbst
zu erweitern und zu bearbeiten, muss man quasi zwingend open-source
Quellcode zur verfügung stellen.
\end{itemize}

\noindent Die grundlegende Idee war die selbe: \emph{Die Entwicklung eines
dezentralen vielseitig einsetzbaren Kommunikationsprotokoll.} Da
allerdings keine einzelne Anwendung angestrebt wurde, ging es
stattdessen um die Entwicklung eines vollständigen Ökosystems und
allgemein einsetzbare Komponenten.\\

\noindent Im nächsten Abschnitt sollen einige dieser Komponenten und
die Entscheidungen die zu ihnen geführt haben beschrieben werden. In
einem weiteren Abschnitt sollen dann die Lektionen und Probleme dieses
erster ersten Entwicklungsphase besprochen werden. 
\subsubsection{Shadow}
\label{sec:org2097318}
Zwar übernahm die erste Implementierung des verteilten
Nachrichtensystems, Codename \texttt{Shadow} weniger Funktionen als die
aktuelle Umsetzung, für das System als ganzes war das Programm aber
nicht weniger wichtig. Der Name lässt sich einfach erklären: für
normale Nutzer sollte das interne Netzwerk niemals sichtbar sein und
sie sollte nie direkt mit ihm interagieren müssen, es war also quasi
\emph{im Schatten}. Geschrieben in \texttt{Elixir} und mit einem \texttt{TCP}-Interface konnte
\texttt{Shadow} sich mit anderen Instanzen verbinden und über eine rudimentäre
Implementierung des \texttt{Kademlia}-Systems Nachrichten senden und
weiterleiten. Um neue Verbindungen herzustellen wurde ein speziell
Entwickeltes System mit so genannten \emph{Member-Files} verwendet. Jedes
Mitglied eines Netzwerks konnte eine solche Datei generieren, mit
welchen dann beliebige andere Instanzen beitreten konnten.\\

\noindent Sobald eine Nachricht im System am Ziel angekommen war,
wurde sie über einen \texttt{Unix-Socket} an den nächsten Komponenten im
System, meistens also \texttt{Hunter} weitergegeben. Dies geschah nur, wenn das
einheitlich verwaltete Registrierungssystem für Personen und Dienste,
eine Teilfunktion von \texttt{Hunter}, ein treffendes Resultat lieferte.
Ansonsten wurde der interne Routing-Table verwendet. Dieser bestand
aus einer Reihe von Prozessen, welche selbst auch direkt die
\texttt{TCP}-Verbindungen verwalteten. 
\subsubsection{Hunter}
\label{sec:org01cf75f}
Während \texttt{Shadow} die Rolle des verteilten Routers übernimmt, ist \texttt{Hunter}
der lokale Router. Es geht bei \texttt{Hunter} also nicht darum, Nachrichten an
andere Mitglieder des Netzwerks zu senden, sondern sie an verschiedene
Applikationen auf der gleichen Maschine zu senden. Jedes beliebige
Programm, unabhängig von Programmiersprache \& internen Strukturen
müsste dann also nur das verhältnismässig Protokoll implementieren und
wäre damit in der Lage, mit allen anderen Komponenten zu interagieren.
Anders als \texttt{Shadow} wurde \texttt{Hunter} komplett in Rust entwickelt und liess
sich in zwei zentrale Funktionen aufteilen:
\begin{itemize}
\item Zum einen diente das Programm als Schnittstelle zu einer einfachen
\emph{Datenbank}, in diesem Fall eine \texttt{JSON-Datei}. Dort wurden alle lokal
aktiven Adressen und die dazugehörigen Applikationen gespeichert.
Ein Nutzer der sich beispielsweise über einen Chat mit dem System
verbindet wird dort mit seiner Adresse oder seinem Nutzernamen und
dem Namen des Chats eingetragen. Wenn dann von einem beliebigen
anderen Punkt im System eine Nachricht an diesen Nutzer kommt, wird
der passende Dienst aus der Datenbank gelesen. All dies lief durch
ein \emph{Command Line Interface}, welches dann ins Dateisystem schreibt.
\item Das eigentliche Senden und Weiterleiten der Nachrichten war nicht
über ein kurzlebiges Programm möglich, da dafür längere Verbindungen
existieren müssten. Dafür muss \texttt{Hunter} als erstes gestartet werden,
wobei das Programm intern für jede Verbindung einen dedizierten
Thread startet.
\end{itemize}

Diese klare Trennung der Aufgaben und starke Unabhängigkeit der
einzelnen Komponenten erlaubt ein einheitliches Nachrichtenformat, da
für die einzelnen Komponenten kein Verständnis über andere Komponenten
oder die Verbindungen haben müssen. 
\subsubsection{NET-Script}
\label{sec:org8a50927}
Ein weitere zentrale Komponente des Systems ist eine eigene
Programmiersprache, welche mit starker Integration in das restliche
System das Entwickeln neuer Mechanismen und Komponenten das System
offener machen sollte. Ein einfacher lisp-ähnlicher Syntax sollte das
Entwickeln neuer Programme einfach und vielseitig einsetzbar machen.
\subsubsection{Interface}
\label{sec:orgd7b1235}
TODO: Interface
\subsubsection{Probleme}
\label{sec:orgd10d8b8}
Die oben beschriebene Architektur hat viele verschiedene Vorteile,
allerdings ist sie nicht ohne Probleme. Grundsätzlich geht es bei
jedem Programm darum, Probleme zu lösen. Einer der zentraler Ideen war
die Modularität, welche es Nutzern erlauben soll, die verschiedenen
Komponenten des Systems einfach zu kombinieren. Und auch wenn dieses
Ziel auf einer technischen Ebene erfüllt wurde, so ist die Umsetzung
alles andere als \emph{einfach}. Die Anzahl möglicher Fehlerquellen steigt
mit jedem eingebundenen Komponenten exponentiell an, und wenn
mindestens vier der Komponenten für selbst die einfachsten Demos
benötigt werden, kann nahezu alles schiefgehen. Dazu kommt, dass viele
Fehler nicht richtig isoliert und verarbeitet würden, weswegen sich
die Probleme durch das System weiter verbreiten würden. Während die
Umsetzung also ihre eigentlichen Ziele erfüllt hatte, war sie noch
lange davon entfernt, für tatsächliche Nutzer einsetzbar zu sein.\\

\noindent Trotzdem wurden die beschriebenen Komponenten vollständig
entwickelt, getestet und vorgeführt. Zwar war es umständlich und nur
bedingt praktisch einsetzbar, trotzdem war es aber eine technisch
neuartige, funktionsfähige Lösung für komplexe und relevante Probleme.
Nachdem die erste Entwicklungsphase erfolgreich abgeschlossen wurde,
kam allerdings noch ein weiteres Problem auf, welches die folgenden
Entscheidungen stark beeinflusst hat, und es ist ein Problem welches
sich auf die grundlegende Natur der Informatik zurückführen lässt:\\
Anders als nahezu alle Studienrichtungen, Wissenschaften und
Industrien, werden in der Informatik die gleichen Werkzeuge verwendet
und entwickelt. Wer die Werkzeuge der Informatik verwenden kann, ist
gleichzeitig in der Lage (zumindest bis zu einem gewissen Grad) neue
Werkzeuge zu entwickeln. Diese Eigenschaft erlaubt schnelle
Iterationen und viele, fortschrittliche Werkzeuge, so kommen
gleichzeitig neue Probleme auf:
\begin{itemize}
\item Neue Methoden und Werkzeuge werden mit unglaublicher Geschwindigkeit
entwickelt und verbreitet. Wer also nicht mit den neusten Trends
mithält kann schnell abgehängt werden. Dies macht auch das
Unterrichten besonders schwer.
\item Natürlich werden die Werkzeuge meistens immer besser und schneller,
allerdings kommt es oftmals auch zu einer Spezialisierung. Dies
führt schnell zu immer spezifischeren, exotischeren Lösungen und
unzähligen Unterbereichen und immer kleineren Gebieten. So
beispielsweise der Begriff \emph{dezentrale Datensysteme}, der zwar ein
einzelnes Gebiet genau beschreibt, für Aussenstehende ist er aber
mehrheitlich bedeutungslos und sorgt für mehr Verwirrung als
Aufklärung.
\item Die immer neuen Gebiete und Gruppen können auch schnell zu Elitismus
führen, wodurch es für Anfänger schwer sein kann, Zugang zu finden.
\end{itemize}

\noindent Diese Eigenschaften, besonders bei unseren sehr neuartigen
Ideen und Mechanismen, machten es unglaublich schwer, Aussenstehenden
die Funktionen und Konzepte zu erklären. Ohne Vorkenntnisse über
Netzwerke und Kommunikationssysteme war es nahezu unmöglich, auch nur
die einfachsten Ideen zu erklären oder den Inhalt dieser Arbeit
darzulegen. Und selbst mit grossem Vorwissen liessen sich nur die
absoluten Grundlagen innerhalb absehbarer Zeit erklären, das Erklären
der theoretischen und technischen Grundlagen würde Stunden in Anspruch
nehmen.\\

\noindent Da am Ende dieser Arbeit zwingend eine zeitlich begrenzte
Präsentation vor einem technisch nicht versierten Publikum stehen
würde, mussten nach dieser ersten Entwicklungsphase gewisse Aspekte
grundlegend überarbeitet werden, diesmal mit einem besonderen Fokus
auf die \emph{präsentierbarkeit}.  
\subsection{Präsentation}
\label{sec:org8127b00}
Auch wenn von der ersten Entwicklungsphase viele Konzepte und sogar
einige Umsetzungen übernommen werden konnten, so gab es grundlegende
Probleme, welche nicht ignoriert werden konnten. Es wurde schnell
klar, dass unabhängig aller technischer Fortschritte eine bessere Art
der Präsentation gefunden werden musste. Dabei ist es wichtig, die
technischen Neuerungen und Besonderheiten nicht zu vergessen. Die
Umsetzung der ersten Entwicklungsphase, wie innovativ und attraktiv
sie auch wirken mag, ist noch weit davon entfernt, von Endnutzern
verwendet oder gar angepasst zu werden. Auch wenn manche der Ideen
hier wieder aufgegriffen werden, musste doch ein grösserer Fokus auf
die \emph{Präsentierbarkeit} der Fortschritte gelegt werden. Daher wurde die
Entscheidung getroffen, die Entwicklung in zwei Bereiche zu
unterteilen:
\begin{itemize}
\item Ein möglichst vielseitig einsetzbares Nachrichtensystem basierend
auf den bereits bekannten Prinzipien wird als Bibliothek für die
Anwendungen sowie öffentlich angeboten. Entwickelt in Rust soll
Geschwindigkeit und Sicherheit garantiert werden und möglichst viele
Möglichkeiten bieten, in andere Applikationen eingebunden zu werden.
\item Anwendungen:
\end{itemize}
\section{Produkt}
\label{sec:orgbbe78fd}
\subsection{Actaeon}
\label{sec:orgc69a86b}
\subsection{Orion}
\label{sec:orgbafa0c8}
\subsection{Anwendungen}
\label{sec:orge32add0}
\section{Ausblick}
\label{sec:org561b501}
\subsection{Verifizierung}
\label{sec:org86ea52b}
\subsection{Balance}
\label{sec:org9b92a73}
\subsection{Anwendungen}
\label{sec:org016d5ad}
\end{document}
